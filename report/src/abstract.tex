% \begin{abstract}
%     This document is a model and instructions for \LaTeX.
%     This and the IEEEtran.cls file define the components of your paper [title, text, heads, etc.]. *CRITICAL:\ Do Not Use Symbols, Special Characters, Footnotes,
%     or Math in Paper Title or Abstract.
% \end{abstract}

% \begin{IEEEkeywords}
%     component, formatting, style, styling, insert
% \end{IEEEkeywords}


\begin{abstract}
    Cardiovascular disease remains a global health challenge, requiring diagnosis and treatment to improve patient health and outcomes. Traditional diagnosis methods, while effective, are resource intensive and susceptible to human error. The study demonstrates machine learning techniques to enhance heard disease prediction using the Cleaveland Heard Disease Datset, containing a diverse set of patient demographics, as well as clinical data. We evaluate the performance of three machine learning models: Logistic Regression, Random Forest, and Artificial Neural Networks (ANNs). Each is scored based on predictive performance and interpretability, as well as computational complexity. Initial results demonstrate that Random Forest and Artificial Neural Networks provide better accuracy, while Logistic Regression offers better transparency. The study demonstrates the potential of machine learning to revolutionize healthcare by enabling automation for risk assessment for heart disease.
\end{abstract}


\begin{IEEEkeywords}
    Heart, Disease, Logistic Regression, Random Forest, Artificial Neural Networks (ANNs), Risk assessment
\end{IEEEkeywords}