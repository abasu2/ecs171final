\section{Literature Review}

Machine learning (ML) has emerged as a transformational instrument in healthcare, especially for predictive modeling and the early identification of illnesses. Machine learning algorithms have shown proficiency in analyzing intricate datasets for heart disease prediction, facilitating diagnosis and therapy planning. This research review examines the use of machine learning in predicting heart disease, emphasizing the efficacy of prevalent models, the importance of feature selection, and the incorporation of these methodologies with health datasets such as the Cleveland Heart Disease Dataset \cite{battineni2020diagnosis}.

\subsection{Predictive Models for Cardiovascular Disease}

The advent of machine learning (ML) has been a total game-changer in the medical field, totally altering the course of illness diagnosis and treatment. Due to its capacity to sift through large datasets in search of patterns that can help with early detection, ML algorithms have been the subject of much research in the field of cardiac illness prediction. For this job, several machine learning models have been successful; the most popular methods are Support Vector Machines (SVM), Random Forest, and Logistic Regression \cite{zhou2021machine},\cite{ren2017forest}. These models are well-suited to different areas of heart disease prediction due to their individual capabilities.

\subsection{Logistic Regression in Predicting Heart Disease}
When it comes to binary classification problems, such determining whether a patient has heart disease or not, Logistic Regression is a popular statistical model. In clinical contexts, where it is crucial to understand the link between traits and outcomes, its interpretability and simplicity make it the preferable option \cite{battineni2020diagnosis}. Logistic Regression allows for easy understanding of feature coefficients by modeling the target variable's likelihood as a function of the input characteristics. It is possible to quantify characteristics that are thought to contribute to the probability of a heart disease diagnosis, for instance, age, cholesterol levels, and resting blood pressure. Nevertheless, datasets with sophisticated patterns may not be well-suited to it because of its dependence on linear correlations, which hinders its capacity to capture complex interactions among elements \cite{zhou2021machine}.

\subsection{Random Forest and Its Resilience}
Because of its resilience and capacity to manage nonlinear interactions, Random Forest is another preferred option for the prediction of heart disease. Random Forest is an ensemble learning technique that builds many decision trees and then combines their predictions to reduce overfitting and increase accuracy \cite{ren2017forest}. Interactions between exercise-induced angina and ST-segment depression are only two examples of the complicated patterns that this model is great at capturing. Random Forest also gives feature significance ratings, which are helpful for figuring out which characteristics are best for predicting cardiovascular disease \cite{zhou2021machine}. Research shows that when compared to simpler models, Random Forest regularly achieves better accuracy than Logistic Regression. Problems may arise in settings with limited resources due to its high computing needs and poor interpretability \cite{soman2009machine}.

\subsection{Support Vector Machines for Intricate Patterns}
The famed high-dimensional space performance and nonlinear separation dataset handling capabilities of Support Vector Machines (SVM) have brought them widespread reputation. Support vector machines (SVMs) divide data into classes by projecting it into higher-dimensional spaces using kernel functions. With this skill, SVMs excel in predicting cardiac problems, even when feature connections are complex \cite{zhou2021machine}. A good example of an interaction that may be well-modeled using an SVM is the one involving maximal heart rate reached, age, and thalassemia. Despite SVMs' great accuracy, they need meticulous parameter optimization, including the choice of kernel type and regularization parameter, to get peak performance. Furthermore, they may not be the best choice for big datasets because to the computing intensity they need \cite{soman2009machine}.

\subsection{Significance of Feature Selection}
In medical applications, where datasets often include several features, feature selection becomes even more important for the performance of machine learning models. Dimensionality, interpretability, and computing efficiency may all be improved by selectively using relevant attributes. Clinically relevant traits are crucial for heart disease prediction, according to research. There are many factors that are known to increase the risk of cardiovascular disease. These include advanced age, certain types of chest discomfort, cholesterol levels, resting blood pressure, and maximal heart rate \cite{battineni2020diagnosis}. It is common practice to include these characteristics into prediction models since they stand as recognized risk factors. More characteristics that have been shown to enhance model performance include ST depression and the number of main vessels colored by fluoroscopy \cite{soman2009machine}.
